% !Mode:: "TeX:UTF-8"
\chapter{参考文献}
\label{cha:bib}
参考文献可以直接写在 \texttt{thebibliography} 环境里, 利用 \cs{bibitem} 罗列文献条目.
虽然费点功夫, 但是好控制, 各种格式可以自己随意改写.

本模板推荐使用 BIB\TeX, 样式为 GB/T 7714—2015 《信息与文献-参考文献著录规则》, 是中国的参考文献推荐标准.
国内的绝大部分学术期刊、学位论文都使用了基于该标准的格式.

根据规则第10.2.2条, 正文中引用多著者文献时, 对欧美著者只需标注第一个著者的姓, 其后附“et al.”; 对于中国著
者应标注第一著者的姓名,其后附“等”字. 姓氏与“et al.”“等”之间留适当空隙.

尽管与学校提供的参考文献格式略有不同, 但更为正式.

看看这个例子: \citet{drl_2018} 总结了现状,\citet{narasimhan_language_2015} 提出了新结构, 
关于书的~\cite{tex1989,algebra2000}, 还有这些 \cite{narasimhan_language_2015,
drl_2018,nikiforov2014,BuFanZhou2016:Z-eigenvalues,HuQiShao2013:Cored-Hypergraphs,
KangNikiforov2014:Extremal-Problems,LinZhou2016:Distance-Spectral,
LuMan2016:Small-Spectral-Radius,Nikiforov2017:Symmetric-Spectrum,Qi2014:H-Plus-Eigenvalues}.

如果需要使用上海大学的引用格式, 可以将模板类文件中
\begin{verbatim}
\RequirePackage[numbers,sort&compress,square,super]{natbib}
\end{verbatim}
一行改为:
\begin{verbatim}
\RequirePackage[numbers,sort&compress]{natbib}
\end{verbatim}

有时候一些参考文献没有纸质出处, 需要标注 URL.
缺省情况下, URL 不会在连字符处断行, 这可能使得用连字符代替空格的网址分行很难看.
如果需要, 可以将模板类文件中
\begin{verbatim}
\RequirePackage{hyperref}
\end{verbatim}
一行改为:
\begin{verbatim}
\PassOptionsToPackage{hyphens}{url}
\RequirePackage{hyperref}
\end{verbatim}
使得连字符处可以断行. 更多设置可以参考 \texttt{url} 宏包文档.