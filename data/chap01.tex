% !Mode:: "TeX:UTF-8"
\chapter{模板介绍}
\label{cha:intro}

这是 \shuthesis\ 的示例文档, 基本上覆盖了模板中所有格式的设置. 建议大家在使用模板之
前, 阅读一下 \texttt{shuthesis.pdf} 文档. \shuthesis\ 已经将 \LaTeX 的复杂性尽可
能地进行了封装, 开放出简单的接口, 以便于使用者可以轻易地使用.
 
\section{模板说明}
\shuthesis\ (\textbf{S}hanghai \textbf{U}niversity \LaTeX\ Thesis Template) 
是为了帮助上海大学毕业生撰写学位论文而编写的 \LaTeX\ 论文模板. 模板的开发分为两
个阶段: 版本 v1.x 是由水寿松制作完成的, 基于 CJK 宏包开发和使用 GBK 编码, 可在
\url{http://blog.lehu.shu.edu.cn/shuishousong/A209370.html} (下载链接已损坏) 下载. 

当前版本是 v2.x, 由 ahhylau 制作完成, 基于 XeCJK 宏包开发, 文件使用 UTF-8 编码. 
\shuthesis\ v2.x 使用文学化编程 (Literate Programming), 利用 \texttt{doc/DocStrip} 
将代码和说明文档混合编写, 便于以后的升级和维护. 另外, 作者重新制作了上海大学 
logo 的高清矢量图, 看起来更加美观. v2.0 托管在 \href{https://github.com/ahhylau/shuthesis}{GitHub} 上.

由于原作者 ahhylau 可能已不再维护项目仓库, 长时间没有接收 pull request, 所以从 v2.1 开始, 由 BlueFisher 维护后续版本, 提供持续的更新支持, 也感谢原作者水寿松与 ahhylau 所做的大量工作! 目前 \shuthesis\ 模板的代码托管在 \href{https://github.com/BlueFisher/shuthesis}{GitHub} 上, 如有修改建议或者其他要求欢迎在 \href{https://github.com/BlueFisher/shuthesis/issues}{GitHub Issues} 上提交 issue, 作者会尽快回复. 由于作者能力精力有限, 非常期待有其他上大的 \TeX\ 使用者加入到模板的开发与维护当中来, 不断完善模板.

本模板是以清华大学学位论文模板 \textsc{ThuThesis} 为基础制作的衍生版, 在此对代码的贡
献者表示感谢!


\section{目录内容}
模板的源文件即为研究生毕业论文中使用的模板, 用户可以通过修改这些文件来编辑自己的毕业论文.
\begin{itemize}
\item{main.tex}: 主文件, 包含封面部分和基本设置.
\item{data}: 包含本文正文中的所有章节.
\begin{itemize}
\item{abstract.tex}: 中英文摘要.
\item{denotation.tex}: 主要符号对照表.
\item{chap01.tex}: 第一章内容.
\item{chap02.tex}: 第二章内容.
\item{chap03.tex}: 第三章内容.
\item{chap04.tex}: 第四章内容.
\item{acknowledgement.tex}: 致谢.
\item{publications.tex}: 作者在攻读学位期间公开发表的论文.
\item{projects.tex}: 作者在攻读学位期间所作的项目.
\item{appendix.tex}: 附录.
\end{itemize}
\item{reference/refs.bib}: 存放论文所引用的全部参考文献信息.
\item{clean.bat}: 双击此文件, 可以用来清理 main.tex 在编译之后生成的所有缓存文件, 
如后缀名为~.aux~,~.log~,~.bak~的文件.
\item{make-doc.bat}: 双击此文件, 一键生成用户手册 \texttt{shuthesis.pdf}.
\end{itemize}


\section{模板使用}
\label{sec:first}

本模板在 Windows 10 / Windows 11 和 \TeX Live 2021 下开发, 所使用的宏包均跟进到最新版本. 本模板并
未在其他平台和发行版进行测试, 如 MacOS \& Mac\TeX. 由于历史原因, 目前国内使用 C\TeX\ 
套装的人还是很多. 然而, C\TeX\ 套装自从 2012 年后就不再更新了, 许多宏包已经很老旧了. 
因此从 \shuthesis\ v2.0 开始, 模板不再支持在 C\TeX 套装下使用 (C\TeX\ 2.9.2 及之前
的版本均无法使用). 如果用户需要在 C\TeX\ 下写作, 可使用 \shuthesis\ v1.x. 在 Windows 
系统和 Linux 系统下作者推荐使用 \TeX Live 进行编译; MacOS 系统可使用 Mac\TeX. 










