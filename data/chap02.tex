% !Mode:: "TeX:UTF-8"
\chapter{表格和插图}
\label{chap:table}

\section{表格}
模板中关于表格的宏包有三个: \pkg{booktabs}、\pkg{array} 和 \pkg{longtabular}. 三线表可
以用 \pkg{booktabs} 提供的 \cs{toprule}、\cs{midrule} 和 \cs{bottomrule}. 它们与
\pkg{longtable} 能很好的配合使用.
\begin{table}[htb]
  \centering
  \begin{minipage}[t]{0.8\linewidth}
    \caption[模板文件]{模板文件}
    \label{tab:template-files}
    \begin{tabularx}{\linewidth}{lX}
      \toprule[1.5pt]
      {\heiti 文件名}  & {\heiti 描述}                                        \\\midrule[1pt]
      shuthesis.ins & \LaTeX{} 安装文件, \textsc{DocStrip}.\footnote{表格中的脚注} \\
      shuthesis.dtx & 所有的一切都在这里面.                                        \\
      shuthesis.cls & 模板类文件.                                             \\
      shuthesis.cfg & 模板配置文.                                             \\
      shuthesis.sty & 常用的包和命令.                                           \\
      \bottomrule[1.5pt]
    \end{tabularx}
  \end{minipage}
\end{table}

\section{插图}
论文里插图可使用 \texttt{graphicx} 宏包.
\begin{figure}[!htbp]
  \centering
  \includegraphics[scale=0.3]{shu.pdf}
  \bicaption{上海大学}{Shanghai University}
\end{figure}

\begin{figure}[!htbp]
  \centering
  \includegraphics[scale=0.2]{shulogo.pdf}
  \bicaption{上海大学 logo}{Shanghai University logo}
\end{figure}


\newpage
\section{多子图}
多子图可使用 \texttt{subcaption} 宏包, 并配合.

\texttt{subcaptionbox} 命令:

\begin{figure}[!htbp]
  \centering
  \bisubcaptionbox{上海大学}{Shanghai University}{
    \includegraphics[scale=0.3]{shu.pdf}
  }
  \bisubcaptionbox{上海大学logo}{Shanghai University logo}[.4\textwidth]{
    \includegraphics[scale=0.3]{shulogo.pdf}
  }
  \bicaption{\texttt{subcaptionbox} 多子图}{subcaptionbox}
\end{figure}


\texttt{captionbox} 命令:

\begin{figure}[!htbp]
  \centering
  \bicaptionbox{上海大学logo}{Shanghai University logo}[.4\textwidth]{
    \includegraphics[scale=0.3]{shulogo.pdf}
  }
  \bicaptionbox{上海大学}{Shanghai University}{
    \includegraphics[scale=0.3]{shu.pdf}
  }
\end{figure}